% Uživatelským požadavkem je~dynamické zařízení sloužící například jako identifikátor hráče, nebo jako herní nástroj.
% Mělo by~ale být schopno zastat i~roli statického zařízení, pro případy her na~delších výletech, kde by~bylo nepraktické nosit s~sebou velké zařízení.
% Vyžaduje tedy dostatečnou mobilitu, aby uživateli nepřekáželo v~pohybu.
% Zároveň by~zařízení mělo být co~nejlevnější pro možnost nasazení ve~velkém počtu.
% Z~toho plynou požadavky na~výslednou konstrukci a~velikost zařízení.

% Zařízení by~mělo být schopno komunikovat s~ostatními zařízeními, ať~už statickými nebo dynamickými.
% Potřebuje také světelný výstup pro zobrazování herních stavů a~jednoduchý vstup pro ovládání.

% %~realizace

Z praktických důvodů jsem tomuto zařízení dal jméno Semisemafor.
Idea zařízení totiž částečně vychází z~projektu Semafor \cite{StarySemafor}, který často sloužil podobně jako dopravní semafor, podle kterého dostal své jméno.

Vstup Semisemaforu bude realizován pomocí dvou tlačítek a~šestiosého IMU, pro možnost používání gest.
Světelný výstup bude realizován pomocí dvanácti RGB LED uspořádaných do~kruhu.
Číslo dvanáct bylo zvoleno proto, aby korespondovalo s~hodinami, typicky pro hry, kde probíhá nějaký odpočet.

Aby nebylo nutné starat se~o~baterii, je napájení zajištěno pomocí USB-A konektoru a~malé powerbanky, která se~tak dá~třeba i~snadno vyměnit za~nabitou.

\section{Výběr součástek}
Protože zařízení bude vyráběno u~firmy JLCPCB, je~výhodné využívat součástky, které mají ve~své nabídce.
Dá se~sice zařídit, aby firma osadila i~součástky od~externího dodavatele, ale je~to o~něco složitější a~je tak jednodušší se~tomu vyhnout.

Aby nebylo nutné pro práci se~Semisemaforem a~AHS používat různá prostředí, je~výhodné použít stejný mikrokontrolér nebo alespoň mikrokontrolér ze~stejné rodiny.
Proto byl zvolen mikrokontrolér ESP32-C3-MINI-1, který je~ve srovnání s~ESP32-S3 výrazně levnější a~aplikaci plně dostačuje. 
ESP32-S3 má~stejně jako ESP32-S3 USB periferii, která se~dá využít na~programování kontroleru.
I~tady je~ale problém, že~se tato metoda dá~softwarově narušit a~Semisemafor je~proto vybaven stejným programovacím konektorem jako ESP32-S3 na~AHS.

Protože mám dobré zkušenosti s~LED WS2812B, zvolil jsem na~LED kruh jejich typickou pětimilimetrovou variantu.

IMU na semaforu nemá sloužit pro žádná přesná měření, ale např. pro detekci jednoduchých gest nikoliv pro přesné měření zrychlení, nezáleží proto tolik na~jeho přesnosti.
Při jeho výběru šlo proto primárně o~cenu a~zvoleno bylo LIS2DH12TR \cite{LIS2DH12TR}, které komunikuje po~SPI nebo I2C, ze~kterých jsem zvolil SPI.

Mikrokontrolér ESP32-C3 i~LIS2DH12TR mají rozsah napájecího napětí do~\(3,6~V\) \cite{ESP32C3}\cite{LIS2DH12TR}.
Není proto možné je~napájet přímo z~napětí na~USB, na~kterém je~k~disposici napětí \(5~V\) a~bude tedy potřeba měnič.
ESP32-C3 požaduje zdroj se~schopností dodat \(0,5~A\) \footnote{specifikováno v katalogovém listu \cite{ESP32C3} na straně 20, v~tabulce 10} a~jeho typická spotřeba ze~zkušenosti nepřesáhne \(200~mA\), LIS2DH12TR pak vyžaduje zanedbatelných \(185~\mu A\) (v~závislosti na~vzorkovací frekvenci i~výrazně méně, viz \cite{LIS2DH12TR} strana 17,~tabulka 12).
Považuji proto za~vhodné pro jeho napájení použít LDO.
Z~nabídky JLCPCB jsem proto zvolil LD39200 \cite{LD39200} pro jeho elektrické parametry a~malé pouzdro.

\section{Návrh schématu a~DPS}

Doplnil jsem blokovací kondenzátory dle doporučení výrobců, zpětnovazební dělič k~LDO a~k~tlačítkům jsem připojil kondenzátor proti odskokům.
Také jsem doplnil rezistory připojující pin mikrokontroleru, využívané při startu procesoru, k~napájení resp. k~zemi, podle jejich účelu. 
Dostal jsem tak schéma, umístěné v příloze \ref{Semisemafor-sch-v1}.

\section{Prototypy}

Při testování první verze, byl problém s~neovladatelnými LED.
Výsledkem pokusů o~nastavení barvy bylo jen náhodné rozsvěcování a~zhasínání.
Když jsem připojil osciloskop na~řídící signál diod, obdržel jsem signál \ref{Semisemafor-zvonek} %~TODO: změřit znovu a~dodat heščí graf

\begin{figure}[!h]
  \begin{center}
    % \includegraphics[width=\textwidth]{text/PraktickaCast/img/trampolina.jpg}
    \includegraphics[width=\textwidth]{text/PraktickaCast/img/osci/fejk.png}
  \end{center}
  \caption{Zarušená komunikace s~LED}
  \label{Semisemafor-zvonek}
\end{figure}

Tento problém jsem vyřešil doplněním feritu do~cesty řídícího signálu, abych utlumil vyšší harmonické složky.

Kvůli servisním zákrokům a~konstrukci krytu jsem navíc přidal i~další dvě kontaktní plošky pod USB konektor.
Důvodem doplnění těchto plošek je~volba nerozebíratelného krytu zařízení a~potenciální možnost narušení programování prostřednictvím USB.
Plošky tak umožní dostat zařízení do~správného stavu pro nahrání programu bez nutnosti zničení krytu. 
Výsledné schéma je umístěné v~příloha \ref{Semisemafor-sch-v2} a~z~něj navržená topologie DPS je vyobrazena v příloze \ref{Semisemafor-pcb-v1}.

\section{Mechanická stavba}
Jedním z~podstatných požadavků byla jistá úroveň krytí proti vodě, aby se~zařízení dalo používat i~za deště.
To nutně neznamená úplnou vodotěsnost, dá~se totiž předpokládat, že~zařízení bude používáno v~poloze, kdy otvor pro powerbanku směřuje k~zemi.
Stačí tedy zajistit těsnost proti stékající vodě v~jednom směru.

Navíc se~během testování objevil ještě požadavek na~zpětnou vazbu tlačítka v~podobě jeho kliknutí. 
Uživatel totiž musí vědět, že~tlačítko skutečně stiskl, na~což je~mechanická odezva samotného tlačítka ideální.

Po několika iteracích jsem obdržel výsledný vzhled (viz obrázek \ref{Semisemafor-box-render})
\begin{figure}[!h]
  \begin{center}
    \includegraphics[width=0.7\textwidth]{text/PraktickaCast/img/Semisemafor-BOX-render.png}
  \end{center}
  \caption{Vzhled Semisemaforu}
  \label{Semisemafor-box-render}
\end{figure}

Jako technologii výroby jsem zvolil obyčejný FDM 3D~tisk.
Původně jsem chtěl použít materiál ASA pro jeho UV~odolnost.
Ukázalo se však, že~jsem nebyl schopen navrhnout mechaniku celého zařízení tak, aby byl výsledek dostatečně voděodolný a~zároveň měla tlačítka uspokojivou zpětnou vazbu.
ASA bylo příliš tuhé a~cvaknutí mikrospínače tak utlumilo natolik, že~bylo uživatelem nezaznamenatelné.
Obdobné výsledky jsem měl u~materiálů PETG, PLA, ABS a~několika dalších.
Jediný materiál, u~kterého jsem dosáhl uspokojivých výsledků, byl polypropylen (PP).

Problém tisku polypropylenu je~jeho tepelná roztažnost, za~běžných teplot proto má při tisku silnou tendenci se~kroutit, což značně zesložiťuje jeho tisk.
Jednou s~možností by~bylo celý tisk provádět při teplotě přes \(120^\circ C\), kde začíná probíhat rekrystalizace a~polypropylen se~začíná výrazně smršťovat.
Tato možnost ale nese nutnost použití speciální tiskárny, která umožňuje tiskový prostor vyhřát na~takto vysokou teplotu, a~proto byla zvolena méně spolehlivá ale jednodušší metoda.

Použitá metoda je~založena na~vícemateriálovém tisku, přičemž primární je~užitečný tisknutý objekt z~polypropylenu a~druhý dobře tisknutelný materiál tvoří podpěry a~přítlak.
Aby tak bylo možné vytisknout i~tvar, který nemá vhodná místa pro umístění přítlaku, musí být opatřen technologickými výstupky, které se~po tisku mohou odříznout.

Tiskový model jsem tedy doplnil o~další objekt zajišťující přítlak a~zároveň i~podpěry.
Výsledek je~vidět na~obr.\ref{Semisemafor-box-pritlak}.

\begin{figure}[!h]
  \centering
  \includegraphics[width=0.9\textwidth]{text/PraktickaCast/img/Semisemafor-BOX-pritlak.png}
  \caption{Soustava modelů pro tisk}
  \label{Semisemafor-box-pritlak}
\end{figure}

Aby nebylo nutné pouzdro tisknout na~více dílů, zvolil jsem možnost zatiskávání DPS během tisku.
Pokaždé, když tisk dospěl do~správného bodu, pozastavil se, aby bylo možné vložit elektroniku a~následně pokračoval.

Kromě elektroniky jsem stejným postupem umisťoval i~průhledové ,,sklíčko''.
Z~důvodu zachování odolnost proti vodě, je toto ,,sklíčko'' také z~polypropylenu, aby se~během tisku přivařilo k~okolní hmotě a~vytvořilo tak vodotěsný spoj.

Protože je~DPS semaforu oboustranně osazena, byl při zatiskávání ještě jeden problém.
DPS by se dala vložit zarovnaná plochou substrátu s~aktuální tiskovou vrstvou.
Což by~však znamenalo, že~by tisková hlava mohla narazit do~nějaké z~vystupujících součástek.

Alternativou by bylo zarovnat aktuální tiskovou vrstvu s nejvyšším bodem nejvyšší součástky, což by~ale znamenalo, že~by elektronika nebyla pevně uchycena.
Tento problém jsem proto vyřešil vložkou, která se~před zatištěním přilepí na~spodní stranu DPS a~srovná ji~tak do~roviny.
Vložka navíc umožnila přítomnost dodatečných kontaktních plošek z~druhé strany USB, protože bez ní~by se~tyto plošky vyzkratovali o~stínění USB konektoru.
Za tímto účelem jsem také DPS nechal vyrobit v~tloušťce 0.8mm, aby byly tyto dodatečné plošky lépe kryty. 
DPS opatřena vložkou je~vidět na~obr \ref{Semisemafor-vlozka}.

\begin{figure}[!h]
  \centering
  \includegraphics[width=0.6\textwidth]{text/PraktickaCast/img/Semisemafor-vlozka.png}
  \caption{DPS Semisemaforu opatřena vložkou}
  \label{Semisemafor-vlozka}
\end{figure}

Aby bylo tlačítko dostatečně měkké, mělo dostatečně silnou odezvu a~zároveň bylo odolné vůči vodě,
zvolil jsem tenkou membránu, ze~které vede šoupátko k~tlačítku.
Při stisku membrány je~tak pomocí šoupátka stisknuto i~tlačítko.
Zároveň aby nebylo nutné mačkat přesně na~místo, kde je~uchyceno šoupátko a~aby se~zabránilo možnému sklouznutí šoupátka z~tlačítka,
zesílil jsem střed membrány, takže se~z~membrány stal hmatník po~obvodu uchycený k~tělu Semisemaforu, jak je~vidět na~obr. \ref{Semisemafor-rez-tlacitky}

\begin{figure}[!h]
  \centering
  \includegraphics[width=0.8\textwidth]{text/PraktickaCast/img/RezSemaforem.png}
  \caption{Řez tlačítky}
  \label{Semisemafor-rez-tlacitky}
\end{figure}

Abych ještě zvýšil odolnost proti vodě, přidal jsem na~desku konformní povlak jako poslední vrstvu ochrany proti vlhkosti.
Ve chvíli, kdy se~tak dostane vlhkost dovnitř Semisemaforu, elektronika přesto zůstane částečně chráněna.

Reálná fotografie dvou Semisemaforů je k vidění na obr. \ref{Semisemafor-real}.

\begin{figure}[!h]
  \centering
  \includegraphics[width=0.8\textwidth]{text/PraktickaCast/img/Real-Semisemafor.jpg}
  \caption{Reálný kus Semisemaforu}
  \label{Semisemafor-real}
\end{figure}