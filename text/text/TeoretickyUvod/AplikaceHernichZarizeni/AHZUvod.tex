Outdoorové hry mají většinou nějaký příběh, který se dá vyprávět konkrétními úkoly na stanovištích.
Na některých stanovištích proto musí být lidská obsluha, na jiných ale může být lidská obsluha s příběhového pohledu nežádoucí.
Když má hráč například vyřadit automatický bezpečnostní systém je lidské obsluha stanoviště prostě poslední možnost.
Podobná stanoviště proto bývají realizovány pomoci různých papírků a provázků.
To určitě má své kouzlo ale i tak je u podobného stanoviště vodné mít obsluhu.
Elektronické řešení podobných stanovišť by ale mohlo otevřít úplně nové možnosti.

Položme si tedy otázku jak by takové zařízení mohlo vypadat.
Podstatný fakt je že prakticky všichni u sebe dnes mají chytrý telefon, čehož můžeme využít.
Nemá proto velký význam aby toto zařízení suplovalo funkce telefonu.
Např. grafický výstup typu display proto v podobném zařízení není potřeb, a v tomto směru už odvádí telefon naprosto dostatečnou práci.
Pokud by tedy v rámci hry bylo potřeba například předat hráči nějaký text nebo obrázek, muže jej zařízení poslat uživateli na telefon.
Možnost propojení s telefonem je také velmi významná při nastavování hry.
Díky telefonu totiž zařízení nepotřebuje uživatelské rozhraní přizpůsobené nastavování.

Mohlo by se zdát že herní stanoviště vlastně ani není potřeba a stačila by mobilní aplikace.
Ale přestože je mobil ve hrách velmi dobře využitelný jsou aplikace na které jednoduše vhodný není.
Pokud má hráč například ze stanoviště získat nějaký fyzický objekt, mobil neposlouží.
Pro hráče a ani organizátory také nemusí zrovna komfortní před hrou zařizovat aby všichni kdo to potřebují měli nainstalovaný správný software natož svůj telefon nechávat někde na stanovišti.
Navíc telefon, nebude dejme tomu na stanovišti v lese dobře viditelný atd.
V neposlední řadě jde i o jistý cool efekt, který běžné zařízení jako mobil nebo třeba tablet neposkytne.

Zařízení určené primárně pro outdoorové hry bychom podle mě měli rozdělit na dvě skupiny, statické a dynamické, protože jsou na tyto skupiny kladeny výrazně jiné požadavky.
Dynamická zařízení jsou ta která muže uživatel pohodlně nosit s sebou.
Do dynamických zařízení by se tak dal zařadit právě i telefon, ten však muže být z různých důvodu nevhodný a proto i tato zařízení dává smysl navrhnout specificky pro hry.
Statické zařízení ja naopak zařízení u kterého se nepředpokládá že jej bude hráč nosit s sebou.
Přesto by mělo být jednoduše přenositelné, není účelem postavit nemobilní hřiště, prostě jen není plánováno aby se se zařízením např. běhalo.
Jde tedy o stanoviště které se jednoduše donese na své místo a při hře se sním nehýbe.
Takové zařízen by tedy mělo být dobře viditelné a splnovat požadavky které na něj hra klade.
Požadavky různých her můžou být ale dost rozdílně, častým požadavkem je něco uchovávat a např. po zadání hesla hráči vydat.
Hra ale taky muže vyžadovat aby bylo zařízení sto přehrát nějakou audio nahrávku nebo ji naopak nahrát.
Z těchto duvodu považujeme za vhodnější zařízení rozdělit na základní řídící jednotku která je samostatně funkční a použitelná při hře, ale ke které se dají jednoduše připojit moduly pro konkrétní herní mechaniky.

Z toho plyne otázka jaká funkcionalita je potřebná v základním zařízení?
Asi žádný systém se kterým hráč přímo interaguje není nutný v každé hře.
Jde tedy o to vybrat takové systémy, které svými nároky nepřevýší užitečnost při hrách.
Ze zkušeností považujeme za nejzákladnější systém nějaký světelný výstup, ten dokáže většinu her velmi příjemně ozvláštnit.
Pochopitelně je také většinou nezbytný nějaký uživatelský vstup, na což většinou stačí obyčejná tlačítka.
Problém je ale určit jaké a kolik jich bude potřeba.
Některé hry vyžadují třeba jen jedno ale takové aby se do něj dalo co nejpohodlněji praštit v běhu, protože je zrovna cílem ke stanovišti co nejrychleji doběhnout.
Jiná hra ale může vyžadovat tlačítek víc, ale už není potřeba aby byli tak velké protože hráč při jejich používání nebude tak akční, ale bude třeba zadávat výsledek nějakého logického úkolu.
Univerzálnější je tedy nepoužívat tlačítka ale nějaký systém který se dá softwarově přizpůsobit.
Příkladem muže být dotyková plocha která se dá softwarově rozdělit na různý oblasti sloužící jako tlačítka a i behem hry se tak dá počet tlačítek měnit.




% Na podobných stanovištích je proto vhodnější nasadit nějaké zařízení, které se chová jako příběhem popsaný systém.
% I když to nebude vypadat tak jak by si hráč onen bájný systém z příběhu nepředstavoval jde o menší změnu než při náhradě organizátorem.%záměnu zařízení za zařízení nikoĺiv o člověka za zařízení.


% V herním scénáři může jít například o únikovou hru ve více týmech kde výsledek jednoho týmu zpřístupní novou herní mechaniku druhému týmu.

% AHS vychází ze zařízení BlackBox \cite{BlackBox} z nějž přebírá řadu konceptů.
% Díky tomuto zařízení bylo možné v posledních letech navrhnout a vyzkoušet řadu her které jej využívají.
% Pro příklad bych zmínil hru Vyhybky, ve které existuje nekolik okruhů tratě po které jezdí vlaky do kterých mohou hráči nakládat 


% Automatické stanoviště může navíc do hry vnést nové mechaniky které by lidské obsluha zajišťovala jen s velkým úsilím.
% Jednotlivá stanoviště spolu totiž mohou komunikovat.

% Krom komunikace mezi stejnými zařízeními je možné komunikovat s běžnými zařízeními jako je telefon nebo notebook, ale i herními zařízeními jiného typu.
% Mezi jiná herní zařízení patří např. SemiSemafor což je minimalistické zařízení sloužící primárně pro identifikaci hráče.
% Pomocí podobných zařízení lze využít AHS na hrubou lokalizaci hráčů.
% Například se tak dá implementovat herní prvek "unikající radiace", který má zdroj u AHS a se vzdáleností klesá jeho intenzita tak jak klesá síla signálu.
% Podobný herní prvek by se bez elektroniky dal provozovat jen velmi složitě.
% V neposlední řade je samostatně fungující stanoviště pro hráče zajímavější než lidská obsluha, které je u podobných hrách mnohem běžnější.

% Cílem této práce je tedy navrhnout elektroniky a mechaniky podobného herního zařízení.