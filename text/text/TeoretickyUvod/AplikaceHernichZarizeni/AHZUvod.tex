Zážitkové hry mají často příběh, který se~dá vyprávět konkrétními úkoly na~stanovištích.
Na některých stanovištích proto musí být lidská obsluha, na~jiných ale může být lidská obsluha z~příběhového pohledu nežádoucí.
Když má~hráč například vyřadit automatický bezpečnostní, systém je~technické řešení vhodnější než lidská obsluha. %lidská obsluha stanoviště poslední možnost.
Její realizace je totiž bližší příběhovému popisu a~hráče tak více vtáhne do~příběhu hry.

Podobná stanoviště proto bývají realizována pomocí různých papírků a~provázků.
To určitě má~své kouzlo, ale i~tak je~u~podobného stanoviště vhodné mít obsluhu, která vysvětlí co~a~jak se~tu~dělá.
Elektronické řešení podobných stanovišť ale může otevřít úplně nový svět možností.

Aby bylo možné vytvořit univerzální automatické herní stanoviště, je~potřeba si~nejprve ujasnit, jaké vlastnosti by~takové zařízení mělo mít.
Za tímto účelem popíši několik her, které jsou většinou navrženy bez použití elektroniky a~popíšu, jak by~se tyto hry mohly použitím elektroniky změnit.

\vspace{-3mm}
\section{Hry}
\vspace{-2mm}
\subsection{King of~the Hill \label{KOTH} }
Tato hra je~převzata z~portálu hranostaj.cz \cite{KingOfTheHill}.

Hra se typicky hraje se dvěma týmy které se~snaží obsadit nějaké území.
Uvnitř hracího pole se~nachází obsazovaná oblast, typicky kruh, který se~hráči snaží obsadit tak, že~do~něj vběhnou.
Ve~stejné vzdálenosti od~obsazovaného kruhu leží základny týmů, ze kterých hráči na začátku hry vyrážejí, aby kruh obsadili. 
Hráči po sobě také mohou házet papírové koule, jejichž zásah znamená, že se zasažený hráč musí vrátit do základny, než bude hrát dál.

Aby tým kruh obsadil musí v~něm hráč daného týmu nějakou dobu vydržet bez zásahu a~bez přítomnosti hráčů z~druhého týmu.
V~kruhu nebo alespoň v jeho blízkosti tedy typicky musí být vedoucí, který měří čas hráčům a~rozhoduje, kdy došlo k jeho obsazení.

Je tedy vhodné, aby tento úkol vykonávalo automatické stanoviště, které bude všechny časy měřit samostatně.
Tímto způsobem se~tak zjednoduší vyhodnocování hry, které by rovněž proběhlo automaticky.
Zároveň se mohou stavy oblastí v průběhu hry zobrazovat např. v~základnách obou týmů, které by tak měli přehled o~dění v~poli.
Hra se tak také dá rozšířit o~další herní prvky, například o~možnost získat nějaké bonusy, když tým obsadí území v~určitém čase.
Také mohou přibýt další obsaditelná území, aby týmy musely bojovat na~více frontách a~pro výhru by~musely zabrat více území naráz.

\vspace{-2mm}
\subsection{Špiónské sítě \label{SpionskeSite}}
Tato hra je~převzata z~portálu hranostaj.cz \cite{SpionskeSite} a~je~určena k~hraní na~pozadí jiné akce, typicky letního tábora.

Hra se hraje na pozadí jiné akce, např. táboru jako doplnění v~čase, kdy neprobíhá jiný program.
Hráči jsou na začátku tajně rozděleni do~několika týmů a~jejich cílem je~zjistit, kdo další do jejich týmu patří.
To dělají pomocí nenápadných otázek a~odpovědí jako např. ,,To máme dnes hezky že?'' s~odpovědí ,,Ani ne, je mi trochu zima.''.
Konkrétní kombinaci otázky a~odpovědi dostávají hráči na začátku hry během rozdělení do týmů.
Podle počtu hráčů a~podle případného příběhu se~určí kolik bude týmů.
Pokud některý hráč najde kolegu, získá bod.
Bod získá i~v~případě, kdy odhalí hráče jiného týmu.

Tato hra se dá použitím elektroniky rozšířit.
Například o~úschovu a~předávání důležitých předmětů, třeba klíčů nebo tajných fotografií.
Každý hráč by~tak dostal předmět, který by~musel předat někomu jinému.
Za tímto účelem by~každý hráč měl svoji zamykatelnou přihrádku, která by~se dala otevřít jen po~zadání hráčem nastaveného hesla.
Protože jsou ale všichni špioni a~navzájem se~neznají, hráč přímo neví, komu má~předmět předat.
Má pouze jeho popis a~musí tedy zjistit, kdo to~je.
Body by~pak šlo získat dvěma způsoby, úspěšným předáním objektu a~odcizením cizího objektu.
Hráči tedy mají motivaci tvářit se~jako osoba, které má~jiný hráč předat svůj objekt, aby se~tak dozvěděli heslo k~jeho přihrádce a~mohli mu~jeho předmět odcizit.
Je faktem, že~po úspěšném předání zůstává přihrádce stejné heslo jako předtím, pokud jej tedy dotyčný sám nezmění.
Pokud tak neučiní, riskuje, že~mu jeho předchozí kolega předmět odcizí, protože už~toto heslo zná.

\vspace{-2mm}
\subsection{Než se~čas naplní}
Tato hra je~převzata z~portálu hranostaj.cz \cite{NezSeCasNaplni}.

Tato hra se hraje ve dvou týmech z~nichž je~jeden výrazně menší než druhý.
Menší tým jsou teroristé, kteří někde schovali bombu.
Druhý tým jsou síly dobra, které se snaží bombu najít a~zneškodnit, v~čemž se jim teroristé snaží zabránit.
Hra končí zneškodněním bomby a vítěstvím sil dobra, nebo výbuchem bomby a vítěstvím teroristů.
Ve hře existují tři různé role, terorista, voják a~pyrotechnik.
Pyrotechnik je jako jediný známý od~začátku hry, zatím co ostatní své role skrývají.
Hráči tedy přesně neví, kdo je s~nimi a~kdo ne.
Herním územím je větší oblast, např. celá vesnice s~blízkým okolím.
Protože na takhle velkém území není možné najít schovanou bombu (papírek s~nápisem Bomba) bez nápovědy, mají hráči k~dispozici nějak zašifrovaná vodítka.

Teroristé mohou vytvářet falešné stopy a~vojáky, resp. pyrotechniky, vyřazovat ze~hry.
Vyřadit protihráče ze hry mohou v případě, kdy jsou s~daným hráčem sami tím, že mu vyjeví svou herní roli.
Daný voják nebo pyrotechnik tak vypadává za hry, zatím co jinému teroristovi se nic nestane.
Pokud je však na dohled této akce jiný voják nebo pyrotechnik, vyřazení ze hry postihne útočníka.

V~této hře se elektronika dá využít na~měření času, který zbývá do~výbuchu bomby a~identifikaci jednotlivých povolání.
Úspěšným vyluštěním jedné z~šifer by~tak hráči mohli získat nějaký předmět, který by~jim poskytl výhodu a~to~třeba částečně i~bez jejich vědomí.
Teroristé tak mohou získat třeba zbraň, co~jim umožní vyřadit víc než jednoho protihráče naráz a~vojáci naopak například neprůstřelnou vestu, která je může ochránit před útokem teroristy.
Při deaktivaci bomby by~také mohlo být se~správným vybavením možné, aby pyrotechnik na~dálku naváděl vojáka při zneškodňování bomby.

\vspace{-2mm}
\subsection{Duchové}
Tato hra počítá už~v~základu s~elektronikou a~je na~ní založena.

Ve hře jsou tři typy zařízení: nabíječka, artefakt a~lucernička.
Hráči mají za~úkol nabít pět artefaktů na~určených místech.
K~tomu jim slouží lucernička, kterou má~každý hráč svoji a~nosí ji~s~sebou, dále nabíječka, která je~společná pro všechny a~během hry se~nepohybuje.
Každá lucerna je~schopna uchovat pouze část energie potřebné k~nabití artefaktu. 
Na nabití artefaktu je~tak třeba více nabitých lucerniček.
Lucernička se~nabíjí přiblížením k~nabíječce, stejně tak se~nabíjí i~artefakt z~lucerničky.
Jak artefakt tak lucernička se~časem sama lehce vybije.
Ve~chvíli, kdy lucernička není v~dosahu artefaktu ani nabíječky a~hráč na~ni~stiskne tlačítko, začne se~vybíjet výrazně rychleji.
Při stisku tlačítka se~ale také lucernička rozsvítí, svítí tak hráči na~cestu a~odpuzuje duchy.
Když se~duch dotkne hráče, hráč nakrátko vypadává ze~hry.
Další nebezpečí duchů ale spočívá v~tom, že~mohou vybíjet artefakty i~lucerničky.
Hráči si~tedy musí dát pozor, aby jim duchové nevybili artefakt během jejich cesty k~nabíječce a~zpět.
% Zároveň mohou být na~hřišti i~skrytí duchové, které nehrají organizátoři, ale jde jen o~samostatné zařízení, které někde leží a~způsobuje tak ve~svém okolí vybíjení.
% Hráči tak musí zjistit, kde se~jim lucerničky vybíjejí a~dávat si~na tato místa pozor.

% V~případě, že~by duchy hrál druhý tým hráčů a~ne organizátoři, měli by~mít nějaký regulační mechanizmus.
% Například by~potřebovali ke~své činnosti jiný druh energie, který by~při své činnosti spotřebovávali.
% Zároveň by~ztráceli energii, když na~ně protihráč posvítí, což by jim omezovalo schopnost škodit a~hráčům umožnilo se bránit.

Tuto hru je~vhodné hrát v~co největší tmě, aby hráči potřebovali světlo svých lucerniček.
Tma navíc umožňuje duchům pohybovat se ve skrytu, což výrazně ovlivňuje herní zážitek.