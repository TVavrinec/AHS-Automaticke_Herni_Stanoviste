Outdoorové hry mají často příběh, který se~dá vyprávět konkrétními úkoly na~stanovištích.
Na některých stanovištích proto musí být lidská obsluha, na~jiných ale může být lidská obsluha z~příběhového pohledu nežádoucí.
Když má~hráč například vyřadit automatický bezpečnostní systém je~lidská obsluha stanoviště poslední možnost.
Podobná stanoviště proto bývají realizovány pomocí různých papírků a~provázků.
To určitě má~své kouzlo, ale i~tak je~u~podobného stanoviště vhodné mít obsluhu.
Elektronické řešení podobných stanovišť by~ale mohlo otevřít úplně nový svět možností.

Abych mohl vytvořit univerzální automatické herní stanoviště, je~potřeba si~nejprve ujasnit, jaké vlastnosti by~takové zařízení mělo mít.
Za tímto účelem popíši několik her, které jsou většinou navrženy bez použití elektroniky a~popíšu, jak by~se tyto hry mohly použitím elektroniky změnit.

\vspace{-3mm}
\section{Hry}
\vspace{-2mm}
\subsection{King of~the Hill \label{KOTH} }
Tato hra je~převzata z~portálu hranostaj.cz \cite{KingOfTheHill}.

Nejlépe se~hraje se~dvěma týmy, může se~ale hrát i~s~více. 
Uprostřed hracího pole je~kruh, ideálně na~vyvýšeném místě, o~který se~bude bojovat. 
Oba týmy mají svou základnu, ze~které na~počátku hry vyráží.
Základny jsou na~opačných stranách kruhu a~zhruba stejně vzdáleny.
Bojuje se~šiškami nebo koulemi. 
Když je~někdo vybit, jde zpátky k~základně, aby se~oživil. 
V~základnách jsou všichni nesmrtelní a~ti~kdo do~ní nepatří, do~ní nevstupují.
Hraje se~většinou na~3~kola. 
V~jednom kole zazní 3~signály.\\
\begin{enumerate}
    \item týmy se~rozeběhnou k~prostřednímu kruhu
    \item kruh je~dostupný k~obsazení (asi půl minuty po~signálu 1)
    \item konec kola, hráči se~vrátí na~základnu (týmy si~můžou základny proměňovat, aby to~bylo víc fér)
\end{enumerate}
\vspace{5mm}
Prostřední kruh se~obsadí tak, že~je v~něm pouze jeden tým, nebo alespoň jeden z~týmu. 
Potom z~něj můžou odejít, ale riskují tak, že~ho obsadí někdo jiný. 
Všichni můžou vstupovat do~kruhu.
U~prostředního kruhu je~vedoucí, který má~několik stopek. 
Měří všem týmům kolik času strávili v~kruhu. 
Konec kola je~tehdy, když stráví jeden tým v~kruhu daný čas (obvykle 10-20 minut). 
Kdo vyhraje kolo dvakrát nebo třikrát (podle toho kolik je~týmů a~jak má~hra být dlouhá), vyhrává.

V~této hře by~se elektronika dala vhodně využít na~měření času, který měl každý tým nadvládu nad územím.
Zároveň by~se zjednodušilo vyhodnocení hry, protože by~se všechny časy jednoduše vyhodnotily automaticky.
I~časování jednotlivých kol by~se tak dalo jednoduše automatizovat.
Hra by~se také dala rozšířit o~další herní prvky, například o~možnost získat nějaké bonusy, když tým obsadí území v~určitém čase.
Také by~mohly přibýt další obsaditelná území, aby týmy musely bojovat na~více frontách a~pro výhru by~musely zabrat více území naráz.

\vspace{-2mm}
\subsection{Špiónské sítě \label{SpionskeSite}}
Tato hra je~převzata z~portálu hranostaj.cz \cite{SpionskeSite} a~je~určena k~hraní na~pozadí jiné akce, typicky letního tábora.

Každý správný filmový špión, který má~navázat kontakt s~neznámým kolegou, mu~musí nejprve nonšalatně položit nenápadnou otázku "Máme to~ale chladný večer, že?" a~teprve pokud mu~neznámý kolega odpoví správnou odpovědí "Ano, v~Paříži je~v~této roční době nezvykle chladno.", tak bude vědět, že~mu může plně důvěřovat.
Hráči se~tedy podobně jako špion ocitnou v~nebezpečné době plné agentů a~tajných organizací a~oni musí rozklíčovat, kdo je~s~nimi a~kdo proti nim.
Hraje se~delší dobu, kdy hra probíhá "na pozadí". 
Podle počtu hráčů nebo příběhu se~určí kolik bude tajných organizací. 
Každá organizace má~jednu kódovou otázku s~navazující odpovědí. 
Jednotlivé otázky spolu s~odpovědí se~napíší na~lístečky a~ty se~náhodně rozdají hráčům tak, aby tajné organizace měli přibližně stejný počet členů (vždy alespoň 2).

Otázky a~odpovědi mohou být například
\vspace{5mm}
\begin{itemize}
    \item Co~jsi včera dělal?
    \item Jezdil jsem na~kole, ale rozbila se~mi přehazovačka.
\end{itemize}
\vspace{5mm}

\begin{itemize}
    \item Dneska je~docela teplo, co?
    \item To~jo, jestli to~takhle půjde dál, tak nám vyschne studna.
\end{itemize}
\vspace{5mm}

\begin{itemize}
    \item Jdeš příští týden na~výpravu?
    \item Bohužel ne, musím být doma a~učit se~zlomky.
\end{itemize}
\vspace{5mm}

Z~logiky věci musí být otázka nenápadná, tak aby případný dotázaný špión z~jiné organizace nepojal podezření. 
Pokud bude mít podezření, může dotyčného (nenápadně) nahlásit vedoucímu hry i~spolu s~podezřelou otázkou, a~pokud uhodne, dostane plusový bod.
Odpověď by~také měla být nenápadná, ale dosti specifická, aby ji~špión z~jiné organizace neodpověděl samovolně.
Cílem je~ptát se~lidí nenápadně a~při nalezení spoluagenta z~organizace postupovat dále systematicky a~neptat se~zbytečně stejného člověka dvakrát na~stejnou otázku.
Hra končí po~časovém limitu, nebo pokud každá organizace najde své členy. 
Na konci se~sečtou body za~každý navázaný kontakt a~body za~každého správně nahlášeného podezřelého agenta.
Hru může hrát i~jen několik agentů, kteří se~musí najít v~"davu civilistů", kteří hru nehrají.

V~této hře by~se elektronika dala využít např. na~úschovu a~předávání důležitého předmětu, třeba klíče nebo tajných fotografií.
Každý hráč by~tak dostal předmět, který by~musel předat někomu jinému.
Za tímto účelem by~každý hráč měl svoji zamykatelnou přihrádku, která by~se dala otevřít jen po~zadání hráčem nastaveného hesla.
Protože jsou ale všichni špioni a~navzájem se~neznají, hráč přímo neví komu má~předmět předat.
Má pouze jeho popis a~musí tedy zjistit kdo to~je.
Body by~pak šlo získat dvěma způsoby, úspěšným předáním objektu a~odcizením cizího objektu.
Hráči tedy mají motivaci tvářit se~jako osoba, které má~jiný hráč předat svůj objekt, aby se~tak dozvěděli heslo k~jeho přihrádce a~mohli mu~jeho předmět odcizit.
Je faktem, že~po úspěšném předání zůstává přihrádce stejné heslo jako předtím, pokud jej tedy dotyčný sám nezmění.
Pokud tak neučiní, riskuje, že~mu jeho předchozí kolega předmět odcizí, protože už~toto heslo zná.

% Některým hráčům by~se tady na~začátku předal i~tento předmět a~číslo přihrádky do~které jej mají tajně uložit.
% Po~uložení do~přihrádky by~museli informovat kolegu o~tom kde a~pod jakým heslem je~předmět uložen.
% Heslo k~přihrádce be~se měnilo takže hráč který předmět uložil by~znal jen část kodu a~zbytek kodu by~se vygeneroval na~osobním zařízení kolegy.
% Kolega by~pak musel spojit svoji část hesla s~tou od~kolegy, předmět získat a~stejným způsobem předat dalšímu hráči.
% Zatím co~by se~jeden tým snažil nenápadně přesunout předmět, druhý tým by~se snažil předmět získat nebo alespoň zjistit o~jaký předmět se~jedná.
% Také by~mohl přímo zjistit vstupní kod k~trezoru a~předmět tak získat přímo tam.

% kod k~trezoru, ve~kterém je~onen důležitý předmět.
% Pomocí kodu by~hráči museli nepozorovaně otevřít přihrádku a~získat tak předmět spolu se~zprávou komu ho~má doručit a~částí přístupového kodu k~příští přihrádce.
% Následně by~musely vypátrat kdo je~jeho kolega a~nepozorovaně mu~předat získanou část kodu.
% Kolegovi by~se na~jeho osobním zařízení vygeneroval zbytek kodu a~
% Kolega by~pak měl za~úkol takto předat předmět dalšímu hráči nebo nepozorovaně předat vedoucímu čímž by~tým získal body za~přesun.
% Zatím co~by se~jeden tým snažil nenápadně přesunout předmět, druhý tým by~se snažil předmět získat nebo alespoň zjistit o~jaký předmět se~jedná.
% Také by~mohl přímo zjistit vstupní kod k~trezoru a~předmět tak získat přímo tam.
% Podobných předmětu by~být víc třeba i~víc druhů a~po přesunu jednoho by~se do~hry přidal další.

\vspace{-2mm}
\subsection{Než se~čas naplní}
Tato hra je~převzata z~portálu hranostaj.cz \cite{NezSeCasNaplni}.
\vspace{2mm}

Legenda\vspace{2mm}\\
Blíží se~vteřina zkázy, teroristé schovali bombu a~teď se~jen třesou nedočkáním, až~tlaková vlna smete z~povrchu zemského nenáviděnou lokaci. 
Tuto škodolibou radost jim poněkud překazily mírové jednotky OSN odhodlané výbuchu zamezit. 
Jak to~dopadne? 
Vše je~v~rukách hráčů.

\vspace{2mm}
Příprava\vspace{2mm}\\
Poté, co~skupinka vedoucích připraví a~následně do~herního pole (velkého klidně přes celou obec) rozmístí tři papíry s~šiframi a~jeden s~nápisem "BOMBA" (který je~o~to zajímavější, že~na něj jsou přicvaknuté tři gumičky), nastane chvíle zasvěcení hráčů do~legendy a~pravidel hry. 
Aby věděli, na~které straně stojí, vylosují si~svou roli -~totiž: voják, pyrotechnik (ti jediní své role odhalí) a~terorista.\vspace{2mm}\\
\vspace{2mm}
Doporučený počet postav:
\begin{itemize}
    \item 3 pyrotechnici
    \item 4~a~víc vojáků
    \item počet teroristů cca polovina počtu vojáků
\end{itemize}

\newpage
Hra\vspace{2mm}\\
Ve stejném okamžiku, kdy začíná hra, spouští se~budík odpočítávající hodinu zbývající do~výbuchu bomby. 
V~této době musí vojáci nalézt papír s~nápisem "BOMBA" a~sehnat pyrotechniky na~její deaktivaci. 
Stačí když každý pyrotechnik roztrhne jednu z~přicvaknutých gumiček, což výbušninu zneškodní. 
Jediný pyrotechnik smí s~bombou manipulovat, nehrozí tedy, že~by ji~teroristé přenášeli z~místa na~místo.
Jelikož je~herní pole rozlehlé, nabízí se~zde berličky pro hledající a~to v~podobě stop, neboli šifer. 
Když vojáci dešifrují zprávu, dozví se~něco víc o~místě, kde bombu hledat. 
V~tom tkví také první příležitost pro teroristy. 
Vytvářením falešných šifer se~dají hledači solidně zmást.
Avšak hlavním posláním teroristů není ani tak mystifikace jako eliminace. 
Stačí, aby se~některý octnul s~vojákem, ba~co hůř s~pyrotechnikem, osamotě a~odhalením své role může ukončit účast dotyčného na~hře. 
Důležitá je~však ta~samota.
Je-li na~dohled jakýkoliv neterorista, vražda neplatí a~vyřazení ze~hry postihne útočníka.
Pokud se~mírové misi podaří bombu do~hodiny zneškodnit, vítězí. 
Jsou-li však hledači příliš pomalí, nebo je~zabit pyrotechnik, vítězí teroristé.

V~této hře by~se elektronika dala využít na~měření času, který zbývá do~výbuchu bomby a~identifikaci jednotlivých povolání.
Úspěšným vyluštěním jedné z~šifer by~tak hráči mohli získat nějaký předmět, který by~jim poskytl výhodu a~to~třeba částečně i~bez jejich vědomí.
Teroristi by~tak mohli získat třeba zbraň, co~by jim umožnila zabít víc jak jednoho člověka naráz a~vojáci naopak dejme tomu neprůstřelnou vestu, která by~je mohla ochránit před utokem teroristy.
Při deaktivaci bomby by~také mohlo být se~správným vybavením možné, aby pyrotechnik na~dálku naváděl vojáka při zneškodňování bomby.

\vspace{-2mm}
\subsection{Duchové}
Tato hra počítá už~v~základu s~elektronikou a~je na~ní založena.

Ve hře jsou tři typy zařízení, nabíječka, artefakt a~lucernička.
Hráči mají za~úkol nabít pět artefaktů na~určených místech.
K~tomu jim slouží lucernička, kterou má~každý hráč svoji a~nosí ji~s~sebou, dále nabíječka, která je~společná pro všechny a~během hry se~nepohybuje.
Každá lucerna je~schopna uchovat pouze část energie potřebné k~nabití artefaktu. 
Na nabití artefaktu je~tak třeba více nabitých lucerniček.
Lucernička se~nabíjí přiblížením k~nabíječce a~stiskem tlačítka, stejně tak se~nabíjí i~artefakt z~lucerničky.
Jak lucernička tak artefakt se~časem sama lehce vybíjí a~když hráč stiskne tlačítko ve~chvíli, kdy není v~dosahu artefaktu ani nabíječky, vybije se~výrazně rychleji.
Při stisku tlačítka se~ale také lucernička rozsvítí, svítí tak hráči na~cestu a~odpuzuje duchy.
Když se~duch dotkne hráče, hráč nakrátko vypadává ze~hry.
Hlavní nebezpečí duchů ale spočívá v~tom, že~mohou vybíjet artefakty i~lucerničky a~to velmi rychle.
Hráči si~tedy musí dát pozor, aby jim duchové nevybili artefakt během jejich cesty k~nabíječce a~zpět.
Zároveň mohou být na~hřišti i~skrytí duchové, které nehrají organizátoři, ale šlo by~jen o~zařízení, které někde leží a~způsobuje tak ve~svém okolí vybíjení.
Hráči tak musí zjistit, kde se~jim lucerničky vybíjejí a~dávat si~na tato místa pozor.

V~případě, že~by duchy hrál druhý tým hráčů a~ne organizátoři, měli by~mít nějaký regulační mechanizmus.
Například by~potřebovali ke~své činnosti jiný druh energie, který by~při své činnosti spotřebovávali.
Zároveň by~ztráceli energii, když na~ně protihráč posvítí, aby měli důvod se~tomu vyhýbat.

Tuto hru je~vhodné hrát v~co největší tmě, aby hráči potřebovali světlo svých lucerniček.
Za světla to~nebude mít tu~správnou atmosféru.
