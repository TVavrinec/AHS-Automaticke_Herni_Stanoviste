U některých her je~potřeba, aby měl hráč zařízení, které bude moci nosit s~sebou a~které mu~při hře bude sloužit jako identifikace nebo nástroj pro plnění úkolů.
Takové zařízení by~mělo být co~nejmenší a~co nejlehčí, aby hráče při hře nezdržovalo.
Navíc by~mělo být co~nejlevnější, aby tolik nevadilo, když jej některý hráč třeba ztratí, což se~přeci jen může stát.
Mimo to~toto zařízení musí být schopno zobrazit svůj stav a~převzít od~uživatele jednoduchý pokyn.

% Rozhodli jsme, že toto zařízení bude svůj stav zobrazovat pomocí pěti inteligentních RGB LED a jako vstup mu budou sloužit dvě tlačítka.
% Abychom nemuseli řešit napájení, má toto zařízení USB konektor a je určeno k napájení powerbankou.
% Toto zařízení jsme nazvali SemiSemafor a jeho vzhled je na obrázku \ref{fig:SemiSemafor}.

% \begin{figure}[h]
%     \centering
%     \includegraphics[width=0.8\textwidth]{text/TeoretickyUvod/AplikaceHernichZarizeni/img/1702085190411.jpg}
%     \caption{Zařízení SemiSemafor}
%     \label{fig:SemiSemafor}
% \end{figure}

% \subsection{Využití zařízení SemiSemafor}
% SemiSemafor je využitelný například ve hře s názvem Duchové.
% Nabiječa, artefakt, SemiSemafor = učastnická lucernička
% učastník dojde k nabiječce, nabije ce a u artefaktu předali energii
% lucerničky se samovibijí,
% když zmáčkneč tlačítko mužeš nabijet artefakt ale když uněj nejseš a lucernička se vybije dvakrát rychleji a odpuzuje duchy (svítí u toho bíle)
% duchová mají taky zařízení (Semiho) když drží tlačítko vybijí lucerničku i artefakty v okolí (nesvítí u toho)
% víc lidí muže naráz nabijet
% když duch chytí hráče, hráč si musí na pět minut ze hry, u nabíječky je respoun
% cílem je nabít všech tt artefaktů

% potenciálně by duchové mohli ke své činnosti potřebovat energiiž