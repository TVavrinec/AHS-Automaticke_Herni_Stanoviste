Z~potřeb popsaných her vyplývají požadavky na zařízení.
Tato zařízení lze rozdělit na statická a~dynamická, podle toho, zda je má hráč nosit všude s~sebou nebo s~nimi jen interaguje na nehybném stanovišti.
V obou dvou případech je vhodné mít co možná nejjednodušší metodu vytváření her.
Není tedy vhodné program pro každou hru psát v samostatném projektu v jazyce C.
Proto je potřeba mít nějakou metodu, která umožní vytvářet hry v~nějakém jednodušším jazyce např. v~Pythonu a nebo JavaScriptu.

\section{Dynamická zařízení}
Dynamická zařízení jsou ta, která má hráč nosit s~sebou.
Tato zařízení by tedy měla být co nejmenší a~nejlehčí, aby hráči nepřekáželo při pohybu.
Zároveň by měla být co nejlevnější, aby se dalo nasadit v~dostatečném množství.
Potřebuje také světelný výstup pro zobrazování herních stavů a~jednoduchý vstup pro ovládání. 

\section{Statická zařízení}
Statická zařízení jsou ta, u~nichž nepředpokládám, že je bude hráč nosit s~sebou.
To ovšem neznamená, že mohou být libovolně velká a~těžká, pořád je potřeba, aby bylo snadné je přesunout z~místa na místo.
Stejně jako dynamická zařízení potřebují světelný výstup, aby bylo možno signalizovat herní stav a~reagovat na hráče.
Také je potřeba vstup, na což většinou stačí obyčejná tlačítka.
Problém je ale určit jaké a~kolik jich bude potřeba.
Některé hry vyžadují třeba jen jedno, ale takové, aby se do něj dalo co nejpohodlněji praštit v~běhu, protože je zrovna cílem ke stanovišti co nejrychleji doběhnout jako u~třeba u~hry King of the hill {viz: \ref{KOTH}}.
Jiná hra může vyžadovat tlačítek víc, ale už není potřeba, aby byly tak velké, protože hráč při jejich používání nebude tak akční, ale bude třeba zadávat heslo, jako u~hry Špionská síť {viz: \ref{SpionskeSite}}.
Univerzálnější je tedy nepoužívat tlačítka, ale nějaký systém, který se dá softwarově přizpůsobit.
Příkladem může být dotyková plocha, která se dá softwarově rozdělit na různé oblasti sloužící jako tlačítka a~i během hry se tak dá počet tlačítek měnit.
Další důležitou vlastností je možnost komunikace s~ostatními zařízeními, která do hry přináší novou možnost jak stanoviště propojit a~také pohodlnou metodu jak stanoviště nastavit přes telefon.
V~neposlední řadě je vhodné mít zvukový výstup, který může být použit např. jako potvrzení zadaného hesla, nebo odezva na prostý klik na dotykovou plochu.

Aby ale bylo možné zařízení použít v~různých hrách, je potřeba aby bylo možné ho přizpůsobit konkrétním potřebám.
Z~toho důvodu považuji za vhodné k~základnímu zařízení moci připojit modul pro konkrétní herní mechaniky.
Z~toho tedy plyne diagram \ref{fig:diagram_zanoreni_0}.
\begin{figure}[h]
    \centering
    \includegraphics[width=0.65\textwidth]{text/TeoretickyUvod/AplikaceHernichZarizeni/diagram/zanoreni_0.pdf}
    \caption{Úvodní blokové schéma zařízení}
    \label{fig:diagram_zanoreni_0}
\end{figure}

Co se světelného výstupu týče, na signalizaci různých stavů je vhodné používat různé barvy světel.
% Jaký vzhled by ale měl mít zdroj barevného světla na podobném zařízení?
Jak je vysvětleno v~následující části \ref{VyuzitiTelefonu}, není potřebné suplovat grafický display, za tímto účelem se dá použít propojení s~telefonem.
Informace, kterou zařízení bude často poskytovat, je čas a~směr, např. čas do konce kola nebo směr k~dalšímu úkolu.
Podobné informace se dají elegantně zobrazit na kruhu.
%Protože má být stanoviště pohodlně čitelné z~blízky i~viditelné z~větší vzdálenosti, je tedy otázkou, zda použít jen jeden kruh, tak aby byl dostatečně viditelný, nebo jich použít více. %%TODO: otázka co s~totu otázkou?
Je vhodné zobrazování rozdělit na dva režimy, čtení na dálku a~čtení na blízko.
Pro čtení na blízko je cílem přímá interakce se zařízením, např. u~zadávání hesla.
Čtení na dálku je naopak určeno pro předávání informací hráči, když právě přímo neinteraguje se stanovištěm, např. který tým má zrovna povolený přístup do zařízení.
Proto je vhodné mít kruhů více, aby bylo možné zobrazovat tyto informace na různých kruzích, které mohou navíc být svému účelu přizpůsobeny.
Jeden kruh tak může svítit jen jedním směrem, aby ho hráč viděl celý najednou pro blízkou interakci, zatímco druhý kruh může svítit do všech stran, aby byl vidět z~co nejvíce míst.

Potřeba propojení s~telefonem nám omezuje možnosti co se týče typu bezdrátové komunikace, protože telefony jsou většinou vybaveny Bluetooth a~WiFi.
Také se v~telefonech rozšiřuje NFC, to je však pro tuto aplikaci z~důvodu krátkého dosahu nevhodné.

Posledním systémem, který je třeba zmínit, je zvukový výstup.
Protože většinou stačí jen jednoduchá zvuková odezva, není potřeba plnohodnotný zvukový systém.
Pro hry, které budou potřebovat přehrávat libovolnou nahrávku, může být použit samostatný zvukový modul, případně je možnost nahrávku přehrát přes uživatelův telefon.
V~základním zařízení je proto potřeba jen jednoduchý bzučák, například jako odezva na kliknutí.
Můžeme tedy diagram upravit na \ref{fig:diagram_zanoreni_1}.
\begin{figure}[h]
    \centering
    \includegraphics[width=0.65\textwidth]{text/TeoretickyUvod/AplikaceHernichZarizeni/diagram/zanoreni_1.pdf}
    \caption{Základní blokové schéma zařízení}
    \label{fig:diagram_zanoreni_1}
\end{figure}

Celé zařízení by také mělo být alespoň částečně voděodolné, aby se dalo použít třeba i~za deště.

\section{Využití telefonu \label{VyuzitiTelefonu}}
Podstatný fakt je, že prakticky všichni u~sebe dnes mají chytrý telefon, čehož mohu využít.
Nemá proto velký význam, aby statické nebo dynamické zařízení suplovalo funkce telefonu.
Např. grafický výstup typu display proto v~podobném zařízení není potřeba, a~v tomto směru už odvádí telefon naprosto dostatečnou práci.
Pokud by tedy v~rámci hry bylo potřeba například předat hráči text nebo obrázek, může jej zařízení poslat uživateli na telefon.
Telefon by se tedy dal zařadit mezi dynamická zařízení.
Možnost propojení s~telefonem je také velmi významná při nastavování hry.
Díky telefonu totiž zařízení nepotřebuje uživatelské rozhraní přizpůsobené k~nastavování, ale prostě se vše nastaví z~telefonu.

Někdy by se mohlo zdát, že herní stanoviště vlastně ani není potřeba a~stačila by mobilní aplikace.
Ale přestože je mobil ve hrách dobře využitelný, jsou aplikace, na které jednoduše vhodný není.
Pokud má hráč například ze stanoviště získat fyzický objekt, mobil neposlouží.
Pro hráče ani organizátory také nemusí být zrovna komfortní před hrou zařizovat, aby měli všichni nainstalovaný správný software.
% Telefon také není například na stanovišti v~lese dobře viditelný.
V~neposlední řadě jde také o~jistý "cool efekt", který běžné zařízení jako mobil nebo třeba tablet neposkytne. %%TODO: tohle chce nějak přeformulovat

