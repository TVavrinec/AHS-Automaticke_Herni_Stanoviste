%Automatické herní stanoviště (AHS) je~nástroj pro tvorbu outdoorových her.
Pravděpodobně si~každý z~nás dokáže vybavit nějakou hru, která se~odehrává venku, někde v~lese nebo na~louce.
Podobné hry bývají typické pro letní tábory nebo třeba skauty.
Často se~jedná o~hru s~jasnými pravidly na~přesně vymezeném hřišti jako je~třeba fotbal nebo možná hravější vlajkovaná, kde je~cílem přenést vlajku soupeře na~své území. 
Často jde ale o~hry, které se~odehrávají v~širém okolí a~průběh se~neskládá z~jen jednoho cíle, jako dát gól, ale spíš z~řady samostatných úkolů, které na~sebe navazují.
Tyto hry také mívají méně či~více výrazný příběh, který hráčům vysvětluje, proč právě dělají to~co dělají a~takové hry budu označovat jako outdoorové hry.

Outdoorové hry, bývají často složeny ze~stanovišť, na~kterých mají hráči plnit různé úkoly.
Aby bylo možné tyto úkoly zadat a~vyhodnotit jejich výsledek, je~většinou nutné, aby na~stanovišti byl nějaký organizátor a~stanoviště obsluhoval.
Tyto úkoly jsou ale často poměrně prosté a~není tak problém je~automatizovat, což může organizátory uvolnit k~jiné činnosti.
Outdoorové hry by~navíc znatelně oživila aktivní komunikace mezi stanovištěmi, která by~mohla i~vytvořit prostor pro nové herní mechaniky.

Řada outdoorových her využívá různá podomácku vyrobená zařízení, které někdo z~organizátorů postavil za~účelem konkrétní hry.
Taková zařízení ale autora stojí velké množství času, protože jej musí celé od~základu navrhnout, vyrobit a~pak je~jej schopen obsluhovat jen on.
Navíc je~pak takové zařízení typicky použito jen u~jedné nebo dvou her, po~kterých jej autor buď rozebere, nebo bezpečně uloží někam, kde si~jej náhodou všimne o~deset let později při úklidu.
V~neposlední řadě bývají jakýmsi zlatým hřebem celé akce např. týdenního tábora a~jejich kouzlo je~především v~odlišnosti od~zbytku akce.

Z~těchto důvodů padlo rozhodnutí na~vývoj univerzálního automatického herního stanoviště, které by~se dalo opakovaně použít na~různé hry i~ve větším počtu.
Podstatnou součástí je~pochopitelně i~pokud možno co~nejintuitivnější ovládání, aby uživatele nezdržovalo od~zábavy.


% % Úvod studentské práce, např\,\dots

% % Nečíslovaná kapitola Úvod obsahuje \uv{seznámení} čtenáře s~problematikou práce.
% % Typicky se~zde uvádí:
% % (a) do~jaké tematické oblasti práce spadá, (b) co~jsou hlavní cíle celé práce a~(c) jakým způsobem jich bylo dosaženo.
% % Úvod zpravidla nepřesahuje jednu stranu.
% % Poslední odstavec Úvodu standardně představuje základní strukturu celého dokumentu.

% % Tato práce se~věnuje oblasti \acs{DSP} (\acl{DSP}), zejména jevům, které nastanou při nedodržení Nyquistovy podmínky pro \ac{symfvz}.%
% % \footnote{Tato věta je~pouze ukázkou použití příkazů pro sazbu zkratek.}

% % Šablona je~nastavena na~\emph{dvoustranný tisk}.
% % Nebuďte překvapeni, že~ve vzniklém PDF jsou volné stránky.
% % Je~to proto, aby důležité stránky jako např.\ začátky kapitol začínaly po~vytisknutí a~svázání vždy na~pravé straně.
% % %
% % Pokud máte nějaký závažný důvod sázet (a~zejména tisknout) jednostranně, nezapomeňte si~přepnout volbu \texttt{twoside} na~\texttt{oneside}!