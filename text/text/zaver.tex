\chapter*{Závěr}
\phantomsection
\addcontentsline{toc}{chapter}{Závěr}

V~práci je~popsáno několik outdoorových her, které nevyužívají elektroniku a~následně je~rozebrána možnost jejich rozšíření o~elektroniku.
Navíc~je popsána i~jedna hra, která od~základu s~elektronikou počítá.
Na základě těchto her jsou odvozeny požadavky, které jsou na~elektronická zařízení v~hrách kladeny.
Následně byl proveden návrh a~výroba dvou zařízení, která tyto požadavky plní a~je tak možné je~v~outdoorových hrách nasadit.

Jednodušší zařízení je~určeno k~tomu, aby jej hráč nosil sebou, a~jeho cílem je~být dostatečně malé a~levné, aby jej bylo možné používat při hrách ve~velkém počtu.
Využívá mikrokontrloer ESP32C3 \cite{ESP32C3}, dvanáct ledek WS2812B \cite{WS2812B} a~jako zdroj malou powerbanku.

Druhé zařízení je~určeno k~tomu, aby zastoupilo organizátora na~stanovišti a~umožnilo mu~tak zapojení do~hry jiným způsobem.
Toto zařízení tedy už~nemusí být tak malé ani levné, protože se~nepředpokládá nasazení v~tak velkém počtu a~je potřeba, aby bylo dobře viditelné.
Zařízení je~rozděleno na~základní řídící jednotku a~moduly, které jsou k~základní jednotce připojeny pomocí atypicky použitého UARTu.
Nestandardně je~UART použit pro komunikaci jeden s~více, namísto standardního jeden s~jedním (viz podkapitola \ref{sec:ModulovyKonektor}).
To má~za cíl umožnit připojení více modulů k~jedné základní jednotce bez potřeby přeposílání zpráv skrz moduly.
Zařízení má~už vlastní baterii a~elektroniku, která se~stará o~její nabíjení a~chrání ji~proti podbití a~přebití.
Řídícím mikrokontrolerem je~ESP32S3 \cite{ESP32S3} a~také jsou zde využity LED WS2812B tentokrát ve~větším množství.  ~
Významnou částí základní jednotky je~také tlaková plocha (viz kapitoly \ref{popisTlakovky1}, \ref{popisTlakovky2} a~\ref{popisTlakovky3}), která umožňuje hráčům interagovat se~základním zařízením pomocí doteku a~tlaku.

Pro obě zařízení bylo také nutné vytvořit vhodný obal, který dokáže odolat např. dešti, kterému mohou být za~provozu vystaveny.
Zařízení byla také zprovozněna a~ověřena jejich funkce.